%&pdflatex
\section{Prima di cominciare}
Prima di entrare nel vivo dell'installazione di Debian, sono necessarie alcune raccomandazioni.

Intanto si deve decidere se si vuole installare Debian sul proprio computer fisico, oppure se si preferisce utilizzare una \textit{macchina virtuale} (ad esempio con \texttt{qemu} o \texttt{virtualbox}). Nel primo caso, se si possiede già un sistema operativo (come Windows) si deve decidere se si desidera rimpiazzarlo completamente o se si preferisce affiancare l'installazione di Debian a quella dell'altro sistema (\textit{dual-boot}).

Nel seguito, descriveremo un esempio di installazione in dual-boot con Windows 10. Saranno comunque spiegate le (poche) differenze nella procedura per installare Debian come unico sistema operativo. Per quanto riguarda la macchina virtuale, non c'è alcuna difformità rispetto all'installazione fisica e la procedura è identica.

Nel caso di installazione fisica è fortemente consigliato effettuare un \textit{backup} di tutti i propri dati personali su un supporto di archiviazione rimovibile: anche nel caso di installazione in dual-boot. Per l'installazione in macchina virtuale, non c'è alcun rischio di perdita di dati.
