%&pdflatex
\section{Parliamo di Linux}
Linux è un sistema operativo \textit{libero} e \textit{open-source}: il codice è liberamente disponibile e chiunque può vederlo e modificarlo (il codice è disponibile su \texttt{kernel.org}). La prima versione del \textit{kernel} Linux (il kernel è il programma del sistema operativo che gestisce l'hardware e le sue risorse e viene avviato per primo con all'accensione del computer, subito dopo il software di \textit{bootstrap}) è la \(0.01\) e risale al 1991. È stato sviluppato da Linus Torvalds. Ad oggi esistono numerose \textit{distribuzioni Linux}, ossia sistemi Linux con software preinstallato e configurato.

Gran parte del software di base disponibile in tutte le distribuzioni Linux fa parte del progetto \textit{GNU}, fondato da Richard Stallman (si veda \texttt{www.gnu.org}). Talvolta si parla infatti di \textit{distribuzioni GNU/Linux}.

Linux è uno dei sistemi operativi più utilizzati al mondo: domina nel mercato mobile con Android (il kernel di Android è Linux); nel mondo dei server e dei supercomputer; e nei \textit{sistemi embedded}. Per quanto non abbia il dominio dei sistemi Desktop (dove spopola \textit{Microsoft Windows}) è comunque molto diffuso.

Si dice talvolta che Linux sia solo per le persone esperte, o che sia utile a svolgere soltanto alcuni compiti specifici. In realtà ciò è falso: \textbf{Linux è per tutti}. Linux è, per molti aspetti, anche molto più facile da utilizzare rispetto a Windows o ad altri sistemi. Ciò che fa sembrare Linux complicato è l'abitudine ad un altro sistema operativo, come appunto a Windows. Ma Linux è diverso da Windows, funziona in modo diverso, e va usato in modo diverso. Se si cerca di utilizzare Linux come si utilizza Windows, inevitabilmente si finirà per fallire, e per ritenerlo erroneamente un sistema operativo complicato. Consideriamo un esempio:

In Linux esistono dei grandi \textit{database} di software, chiamati \textbf{repository}, dai quali l'utente può scaricare tutti i programmi di cui ha bisogno. I repository esistono anche nel mobile, e i pochi lettori di questo documento probabilmente li usano tutti i giorni senza saperlo: il \texttt{Play Store} è il repository di Android; l'\texttt{App Store} è il repository di iOS per iPhone e iPad. Quindi in Linux, si può dire, il software si installa come si installerebbe su uno smartphone: da un'app dedicata che funziona da \textit{shop}.

In Windows funziona in un altro modo: il software va scaricato dal web e installato. Per installare un software su Windows bisogna seguire questi passaggi:
\begin{enumerate}
	\item Cercare su Google (o un altro motore di ricerca) il nome del programma;
	\item Scegliere il sito giusto (altrimenti \textit{malware} e \textit{virus} a volontà!);
	\item Navigare in mezzo al sito pieno di pubblicità per trovare il download;
	\item Scaricare l'\textit{installer};
	\item Avviare l'installer;
	\item Disattivare tutta la pubblicità e il software che non si desidera dall'installer;
	\item Configurare l'installazione (Avanti, Avanti, Avanti, \ldots);
	\item Installare.
\end{enumerate}
In Linux, invece (presentiamo un esempio con Debian, in altre distribuzioni può essere leggermente diverso), per installare un programma è sufficiente:
\begin{enumerate}
	\item Scrivere nel terminale: \texttt{apt-get install \textit{nome-prog}}
\end{enumerate}
Fine. Tutto qui. Le persone, quando provano Linux, finiscono spesso per tentare di installare programmi scaricandoli dal web. Siccome non ci riescono (perché questo non è il modo in cui in Linux si scaricano e si installano i programmi) allora finiscono per tornare su Windows pensando che Linux sia complicato. Ma Linux non è complicato, anzi è pure più semplice (un solo passaggio contro otto passaggi) --- però è \textit{diverso}.

È normale incontrare difficoltà all'inizio con Linux: questione di abitudine, il lettore non si scoraggi alla prima difficoltà.
